% Options for packages loaded elsewhere
\PassOptionsToPackage{unicode}{hyperref}
\PassOptionsToPackage{hyphens}{url}
%
\documentclass[
  12pt,
]{article}
\usepackage{amsmath,amssymb}
\usepackage{lmodern}
\usepackage{iftex}
\ifPDFTeX
  \usepackage[T1]{fontenc}
  \usepackage[utf8]{inputenc}
  \usepackage{textcomp} % provide euro and other symbols
\else % if luatex or xetex
  \usepackage{unicode-math}
  \defaultfontfeatures{Scale=MatchLowercase}
  \defaultfontfeatures[\rmfamily]{Ligatures=TeX,Scale=1}
\fi
% Use upquote if available, for straight quotes in verbatim environments
\IfFileExists{upquote.sty}{\usepackage{upquote}}{}
\IfFileExists{microtype.sty}{% use microtype if available
  \usepackage[]{microtype}
  \UseMicrotypeSet[protrusion]{basicmath} % disable protrusion for tt fonts
}{}
\usepackage{xcolor}
\usepackage[margin=1in]{geometry}
\usepackage{graphicx}
\makeatletter
\def\maxwidth{\ifdim\Gin@nat@width>\linewidth\linewidth\else\Gin@nat@width\fi}
\def\maxheight{\ifdim\Gin@nat@height>\textheight\textheight\else\Gin@nat@height\fi}
\makeatother
% Scale images if necessary, so that they will not overflow the page
% margins by default, and it is still possible to overwrite the defaults
% using explicit options in \includegraphics[width, height, ...]{}
\setkeys{Gin}{width=\maxwidth,height=\maxheight,keepaspectratio}
% Set default figure placement to htbp
\makeatletter
\def\fps@figure{htbp}
\makeatother
\setlength{\emergencystretch}{3em} % prevent overfull lines
\providecommand{\tightlist}{%
  \setlength{\itemsep}{0pt}\setlength{\parskip}{0pt}}
\setcounter{secnumdepth}{5}
\newlength{\cslhangindent}
\setlength{\cslhangindent}{1.5em}
\newlength{\csllabelwidth}
\setlength{\csllabelwidth}{3em}
\newlength{\cslentryspacingunit} % times entry-spacing
\setlength{\cslentryspacingunit}{\parskip}
\newenvironment{CSLReferences}[2] % #1 hanging-ident, #2 entry spacing
 {% don't indent paragraphs
  \setlength{\parindent}{0pt}
  % turn on hanging indent if param 1 is 1
  \ifodd #1
  \let\oldpar\par
  \def\par{\hangindent=\cslhangindent\oldpar}
  \fi
  % set entry spacing
  \setlength{\parskip}{#2\cslentryspacingunit}
 }%
 {}
\usepackage{calc}
\newcommand{\CSLBlock}[1]{#1\hfill\break}
\newcommand{\CSLLeftMargin}[1]{\parbox[t]{\csllabelwidth}{#1}}
\newcommand{\CSLRightInline}[1]{\parbox[t]{\linewidth - \csllabelwidth}{#1}\break}
\newcommand{\CSLIndent}[1]{\hspace{\cslhangindent}#1}
\usepackage{dcolumn}
\usepackage{float}
\floatplacement{figure}{ht}
\usepackage{setspace}
\usepackage{fancyhdr}
\pagestyle{fancy}
\usepackage{lastpage}
\setlength{\headheight}{14.49998pt}
\addtolength{\topmargin}{-2.49998pt}
\usepackage{rotating, graphicx}
\renewcommand{\topfraction}{.85}
\renewcommand{\bottomfraction}{.7}
\renewcommand{\textfraction}{.15}
\renewcommand{\floatpagefraction}{.66}
\setcounter{topnumber}{3}
\setcounter{bottomnumber}{3}
\setcounter{totalnumber}{4}
\usepackage[hang,flushmargin]{footmisc}
\usepackage{lipsum}
\usepackage{etoolbox}
\usepackage{mlmodern}
\usepackage[T1]{fontenc}
\usepackage{enumitem}
\setlist{noitemsep}
\AtBeginEnvironment{quote}{\singlespace\vspace{-\topsep}\small}
\AtEndEnvironment{quote}{\vspace{-\topsep}\endsinglespace}
\providecommand{\keywords}[1]{\textbf{Keywords:} #1}

\usepackage{booktabs}
\usepackage{longtable}
\usepackage{array}
\usepackage{multirow}
\usepackage{wrapfig}
\usepackage{float}
\usepackage{colortbl}
\usepackage{pdflscape}
\usepackage{tabu}
\usepackage{threeparttable}
\usepackage{threeparttablex}
\usepackage[normalem]{ulem}
\usepackage{makecell}
\usepackage{xcolor}
\ifLuaTeX
  \usepackage{selnolig}  % disable illegal ligatures
\fi
\IfFileExists{bookmark.sty}{\usepackage{bookmark}}{\usepackage{hyperref}}
\IfFileExists{xurl.sty}{\usepackage{xurl}}{} % add URL line breaks if available
\urlstyle{same} % disable monospaced font for URLs
\hypersetup{
  hidelinks,
  pdfcreator={LaTeX via pandoc}}

\title{The Myth of the Myth of Independence:\\
A Critique of Binder and Spindel's Appraisal of the Fed}
\author{Zachary Thomas McDowell\\
BA \& MA in Political Science\\
University of Georgia}
\date{\textbf{Updated}: July 31, 2022}

\begin{document}
\maketitle
\begin{abstract}
\singlespace The response to Desmond King's review of \emph{The Myth of
Independence} by one of the book's authors reveals one of two possible
critical flaws in their thesis that the relationship between Congress
and the Federal Reserve is best described as one of interdependence.
Binder fails to adequately address the most important of King's
critiques. This article expands on the points raised by King in a more
comprehensive manner than a book review allows for, with a particular
emphasis on contributing novel empirical analyses that cast further
doubt on the arguments positied by Binder.
\end{abstract}

\keywords{Inequality, Federal Reserve, Monetary Politics, Financial Crisis}

\let\thefootnote\relax\footnotetext{\lipsum[1][1-7]}

\thispagestyle{empty}

\newpage
\fancypagestyle{TOC}{%
    \fancyhead[R]{}
    \fancyhead[L]{}
}

\thispagestyle{TOC}
\tableofcontents
\listoffigures
\listoftables
\clearpage
\newpage

\fancypagestyle{INTRODUCTION}{%
    \fancyhead[R]{}
    \fancyhead[L]{}
}

\thispagestyle{INTRODUCTION}
\clearpage

\spacing{1.5}

\hypertarget{introduction}{%
\section*{Introduction}\label{introduction}}
\addcontentsline{toc}{section}{Introduction}

In 2018, the journal \emph{Perspectives on Politics} published an
interesting back-and-forth between two opposing sets of monetary
politics scholars. Sarah Binder and Mark Spindel are the authors of
\emph{The Myth of Independence}, which argues that the Federal Reserve
has an ``interdependent relationship'' with Congress. Both institutions
rely on each other for different things, so neither is fully dependent
on nor independent from the other, hence why Binder and Spindel refer to
their relationship as interdependent. On the other hand, Desmond King
and Lawrence Jacobs argue in \emph{Fed Power} that the Federal Reserve
is effectively independent from Congress and holds more institutional
power than it has any point in its history. They argue that the Fed's
policies, under normal economic conditions, serve to worsen economic
inequality by favoring financial capital over working-class Americans.
The \emph{Perspectives on Politics} article is unique in that it is
structured as follows:

\begin{enumerate}
\def\labelenumi{\arabic{enumi}.}
\tightlist
\item
  King reviews \emph{The Myth of Independence}
\item
  Binder and Spindel respond to King's review
\item
  Binder and Spindel reviews \emph{Fed Power}
\item
  King and Lawrence respond to Binder and Spindel's review
\end{enumerate}

In their response, King and Lawrence summarize their main point of
disagreement with \emph{The Myth of Independence}, stating that ``Sarah
Binder and Mark Spindel view the Fed's actions as part of a dance with
Congress that is largely silent about the winners and losers outside of
Washington. By contrast, Fed Power puts the distributional consequences
of the central bank's policy front and center, along with the politics
that produces them'' (\protect\hyperlink{ref-binder2018c}{Binder et al.
2018}, pg. 783). In other words, Binder and Spindel do not pay
sufficient attention to the distributional consequences concerning
income and wealth that are the result of Fed policies, nor do they
appear to recognize the real amount of power held by the Fed.

Moreover, King and Lawrence, in their joint-response, list their top
four criticisms of \emph{The Myth of Independence}, wherein they posit
that Binder and Spindel:

\begin{enumerate}
\def\labelenumi{\arabic{enumi}.}
\tightlist
\item
  Overstate the Fed's ``deference to Congress and under appreciates the
  Fed's will and capacity to evade legislative control''
  (\protect\hyperlink{ref-binder2018c}{Binder et al. 2018}, pg. 783).
\item
  Under appreciate the fact that the Fed ``is independent of the
  congressional budget appropriations process''
  (\protect\hyperlink{ref-binder2018c}{Binder et al. 2018}, pg. 783).
\item
  Omit any discussion of the phenomenon known as ``financialization''
  (\protect\hyperlink{ref-binder2018c}{Binder et al. 2018}, pg. 783).
\item
  Fail to recognize the clear institutional bias towards the wealthy,
  which can can be seen in how ``the Fed's selective benefits for finance
  and the conduct of monetary policy produce clear winners among the
  most affluent'' (\protect\hyperlink{ref-binder2018c}{Binder et al.
  2018}, pg. 783).
\end{enumerate}

What bothers me, as a scholar, is how Binder and Spindel egregiously
failed to refute King and Lawrence on the actual terms he presented,
almost as if they there will intentionally ignoring the crux of King's
criticisms because they had no way to counter him directly, instead
relying on obfuscation to defend their claims. Binder and Spindel
responded in such a way as make it seem that they were avoiding the
critiques raised in a seemingly condescending way.

Before we can begin arguing about whether or not the Fed is dependent on
Congress, we must first define what me mean by ``dependent.'' Further,
we must also pin down the contextual and effective meaning of
``accountability'' when discussing the relationship between the two
institutions. In my view, passing legislation is but one part of the
matrix of components that constitutes what is necessary, but not
sufficient, to hold the Fed accountable. I'm more interested in
investigating whether or not it can be reasonably argued that the Fed,
by virtue of its dependency on Congress, has been \emph{meaningfully}
held accountable by Congress for its various failings.

Without descending into a pedantic discussion of how to best understand
and define accountability in its totality, my argument relies on the
incorporation of one critical element involved in an entity being held
accountable: for something to be held accountable, there must be at
least some longevity in the newly applied constraints. In the context of
the Fed, this would mean that any reforms passed by Congress would
effectively constrain a particular set of dissatisfying actions taken by
the Fed for the indefinite future, up to and until such reforms are
legally undone by Congress.

\hypertarget{stasis-issues}{%
\section{Stasis Issues}\label{stasis-issues}}

It's obvious that the authors in question are not addressing each other,
thus creating stasis issues in this ``critical dialogue.'' It is
frustrating to watch accomplished and no doubt knowledge Fed scholars
talk past each other, but nonetheless here we are. What follows is a
systematic breakdown of the myriad stasis issues I identified in my
reading of the critical back-and-forths presented in Binder et al.
(\protect\hyperlink{ref-binder2018c}{2018}).

\hypertarget{inequality-the-fed}{%
\subsection{Inequality \& the Fed}\label{inequality-the-fed}}

The most egregious example of Binder totally missing the point is how
she responds to King's assertion that the Fed's post-2008 policy led to
higher levels of income inequality. To highlight the degree to which
Binder misses King's point, here are their respective words:

\begin{quote}
\textbf{King}: Undeniably, the Fed's interventions after the 2008 Great
Recession to prevent a collapse of the financial system in the United
States and globally spared many from job loss and misery\ldots{} it is
important to avoid a false equivalency between the gains for finance and
those for the general public. The Fed's policies delivered lopsided and
often concealed benefits for finance, the top 1\%, and the institutional
interests of the Federal Reserve Bank that enjoys more power and
autonomy than at any time in its 100-year history
(\protect\hyperlink{ref-binder2018c}{Binder et al. 2018}, pg. 780).
\end{quote}

\begin{quote}
\textbf{Binder}: Congress authorizes the Fed to make emergency loans to
banks---not to steer aid directly to homeowners. In a crisis, monetary
policy can affect the real economy by pumping credit through the clogged
plumbing of the financial system. And evidence from progressive
economists suggests that the Fed's unconventional bond purchases reduced
mortgage rates, making working- and middleclass Americans better-off
(\protect\hyperlink{ref-binder2018c}{Binder et al. 2018}, pg. 781).
\end{quote}

I've done my best to fairly and accurately represent each authors'
views, while still being concise. If anyone has issues with how I've
represented the above exchange, I welcome any and all criticisms, given
the subjective nature of how I chose to shorten the respective quotes.
That said, insofar as the above quotes \emph{do} accurately and fairly
represent the authors' respective views concerning the Fed's role in
contributing to the 2008 financial crisis and the subsequent worsening
of income and wealth inequality, on top of the aforementioned job losses
and misery, Binder's response is lacking, to say the least.

Pointing to the fact that Fed's bond buying supposedly ``reduced
mortgage rates, making working- and middleclass Americans better-off''
is tantamount to lauding an engineer for filling a single crack on a
fully broken dam. Binder's response implicitly under-emphasizes the
apocalyptic experiences felt by millions of Americans and citizens in
countries the world over. There may not be enough time in this article
to discuss the Fed's role in propping up the European Union during the
Eurocrisis, but its at least worth mentioning for the time being. For
now, it is necessary to specify what we mean when discussing and
subsequently operationalizing ``economic inequality.''

The most common way scholars measure \emph{income} inequality is by
calculating a country's Gini coefficient, which is a index value ranging
from 0 to 1, with 0 representing perfect equality and 1 representing
perfect inequality. The World Bank provides the following, more
comprehensive explanation:

\begin{quote}
The Gini index measures the extent to which the distribution of income
(or, in some cases, consumption expenditure) among individuals or
households within an economy deviates from a perfectly equal
distribution\ldots{} The Gini index measures the area between the Lorenz
curve and a hypothetical line of absolute equality, expressed as a
percentage of the maximum area under the line. Thus a Gini index of 0
represents perfect equality, while an index of 100 implies perfect
inequality (\protect\hyperlink{ref-worldbank2022}{World Bank 2022}).
\end{quote}

\autoref{income_dist1} shows that, contrary to what Binder and Spindel
seem to imply, inequality in the US worsened post-2008. And to bolster
King and Jacobs's argument that the Fed should have emulated the Bank of
Canada, \autoref{income_dist1} also shows that the income share of their
Top 1\% decreased sharply from 2008 to 2013, whereas the income share
going to the Top 1\% in the US \emph{increased} sharply post-2008. But
where does the Fed come in? According to Leonard---and in stark contrast
to the claims of Binder and Spindel---the Fed's policy of
``quantitiative easing'' (QE) did more than any other single policy ``to
widen the divide between the rich and the poor
(\protect\hyperlink{ref-leonard2022}{Leonard 2022}, pg. 10).

\begin{figure}[ht]

{\centering \includegraphics{fed-paper_files/figure-latex/income_dist1-1} 

}

\caption{\label{income_dist1}Share of National Income (\%)}\label{fig:income_dist1}
\end{figure}

To critique King, he could have done a better job of demonstrating his
claims concerning the Fed and the 2008 financial crisis by providing
citations and at least a few more lines of discussion. But it is unclear
if such improvements on King's part would have elicited a different
response from Binder, given that the crux of their disagreement has
nothing to do with the factual validity of each others' claims. Instead,
they are effectively discussing two completely different aspects of the
financial crisis that can simultaneously be true. My issue with Binder's
response has nothing to do with the factualness of her claim, rather
that she 1) underplays the severity of the crisis and 2) makes it seem
that the Fed's bond buying \emph{reduced} inequality rather than
increase it.

\hypertarget{independence-vs.-interdependence}{%
\subsection{Independence
vs.~Interdependence}\label{independence-vs.-interdependence}}

The thing that jumps out from Binder's response to King is their
remarkable ignorance, whether it is willful or genuine, of the core
content of King's critiques, especially on the topics of income
inequality and the 2007-08 financial crisis. Let me be clear, so that
scholars like Binder and Spindel have no excuse for misunderstanding or
misrepresenting my position: The Federal Reserve is fully independent in
all meaningful areas when it comes to its congressionally mandated
responsibilities. I disagree with Binder and Spindel's argument that
simply observing that there are barely more than a handful of instances
since 1913 when Congress has ``stepped in'' to recalibrate the Fed is
sufficient evidence to confidently assert that the Fed is, to a large
degree, \emph{dependent} on Congress due to a myriad of political
concerns held by legislators. Rather, concurring with one of King's
critiques, Binder and Spindel completely ignore the very obvious fact
that the Fed lobbies Congress for all sorts of things, not dissimilar to
the behavior of private firms and special interest groups. By ignoring
the mere capacity for active lobbying by the Fed, Binder and Spindel
bias their observations in their favor. They treat every legislative
action that concerns the Fed as totally external to and uninfluenced by
the Fed itself, which serves to reinforce Binder and Spindel's idea that
the Fed is a mere subject of Congress and not an institution that
frequently interacts and coordinates with Congress to achieve its own
goals.

In my view, again concurring with King, \emph{the} critical deficiency
in Binder and Spindel's analysis is their frequent retreat ``to a
troubling form of argument about an `intuitive' (p.~237) sense that the
Fed is dependent on Congress, rather than a cogent demonstration of this
proposition'' (\protect\hyperlink{ref-king2018b}{King 2018}, pg. 780).
No amount of pontificating about the very real pieces of legislation
that have changed the parameters of the Fed's institutional capacity as
a central bank is sufficient to demonstrate that the Fed truly, in any
meaningful way, is \emph{dependent} on Congress. Instead, they seem to
find it sufficiently compelling to draw the reader's attention to the
mere fact that Congress has passed laws that, on their face, succeed in
holding the Fed \emph{accountable}. But how, with a straight face, can
someone say that the Fed has been effectively reigned in by legislation,
has been successfully held accountable by Congress? As stated earlier,
the Fed is the unequivocal most powerful central bank in the world,
despite its history of drawing the ire of American voters, their elected
representatives, and untold millions of others around the world that
have felt, and will nevertheless continue feel, the effects of Fed
policy.

King cites Donald Kettl's \emph{Leadership at the Fed} as a complement
to Binder and Spindel's book, wherein both works discuss the historical
dysfunctionality of Congress as an institution. That said, King asks:
How are we to believe that such a dysfunctional institution as Congress
is at all capable of regulating the Fed in any \emph{meaningful} sense?
Further, how can it be said that the Fed has been effectively held
accountable when it is more powerful now than at any point in its
history?

The main point to make is that Congress is barely capable of handling
mundane legislative business, so its a bit of a stretch to give it the
benefit of the doubt concerning its capacity to effectively control the
Fed. Could it be any other way? The Fed is vastly more technically
advanced in its operations and communications than Congress. Just look
at the difference in budgets between the two institutions and it will
become clear which one holds the advantage when it comes intellectually
combating the other.

\hypertarget{transparency-vs.-accountability}{%
\subsection{Transparency
vs.~Accountability}\label{transparency-vs.-accountability}}

Binder appears to equate legislation requiring increased transparency at
the Fed as demonstrable proof that Congress is fully capable of holding
the Fed accountable for its actions, especially its failures. To pull
from an earlier quote by King, the Fed ``enjoys more power and autonomy
than at any time in its 100-year history''
(\protect\hyperlink{ref-binder2018c}{Binder et al. 2018}). So the
question we should be asking ourselves is: How can it be said that the
Fed has been held accountable by Congress throughout its 100 year
existence despite it currently wielding more power than ever? How does
an institution accrue such power while being under the control of
Congress?

\hypertarget{a-peoples-history-of-the-fed}{%
\subsection{A People's History of the
Fed}\label{a-peoples-history-of-the-fed}}

The scholarly treatment of the history of the Fed is one of the most
curious phenomena in the social sciences. As Binder and Spindel did in
\emph{The Myth of Independence}, and as countless others have done and
will continue to do, the Fed's essential history is well documented.
That said, the \emph{essential} history of the Fed is one of the
obstacles that prevents most scholars from engaging with the Fed in
recognition of what it is, that being the single most powerful and
influential central banks in the world, an institution that garners the
attention of both the finance minsters of every country as well as every
bank and financial asset manager.

\begin{figure}[ht]

{\centering \includegraphics{fed-paper_files/figure-latex/fed_funds-1} 

}

\caption{\label{fed_funds_rate}Effective Federal Funds Rate (\%)}\label{fig:fed_funds}
\end{figure}

One need not look far back into 20th century history to observe the
Fed's actions affecting the world economy, not to mention the fact that
one could argue that the European Union owes its post-2015 existence to
the Fed's dollar swap-lines.

\hypertarget{original-empirical-analysis}{%
\section{Original Empirical
Analysis}\label{original-empirical-analysis}}

What causes inequality? More specifically, what causes income
inequality? According the Binder and Spindel, not the Fed! Not the Fed's
standard operating procedures! I'm more inclined to believe King and
Jacobs when they say, on the contrary, that the Fed is an inequality
machine, with its each of typical policies and actions serving to worsen
inequality in the US and the world. But which set of scholars is right?

To see which book's thesis holds up better under scrutiny, I have
created an empirical model designed to capture all of the major
contributors to income inequality. The first question of interest
concerns the role of financial capital assets in worsening income
inequality.

\hypertarget{literature-review}{%
\subsection{Literature Review}\label{literature-review}}

There is a large body of literature that examines the various social and
political phenomena caused by inequality, but less work has been done to
explain what causes inequality in the first place.

\hypertarget{hypotheses}{%
\subsection{Hypotheses}\label{hypotheses}}

\begin{itemize}
\item
  \textbf{H1}: As financial assets \emph{increase} as a share of GDP,
  income inequality will \emph{increase.}
\item
  \textbf{H2}: As union density \emph{increases}, income inequality will
  \emph{decrease.}
\item
  \textbf{H3}: As the real rate of return on capital \emph{increases},
  inequality will \emph{increase.}
\end{itemize}

\hypertarget{methods}{%
\subsection{Methods}\label{methods}}

Following the recommendations found in
(\protect\hyperlink{ref-beck1995}{\textbf{beck1995?}}), I have employed
a time-series panel study of 24 OECD countries.

\hypertarget{results}{%
\subsection{Results}\label{results}}

\autoref{ineq1} shows, across multiple model specifications, that a one
unit increase in financial assets (as a share of GDP) is associated
with, on average and all things being equal, between a 0.204 and 1.346
unit increase in income inequality, depending on how inequality is
measured.

\begin{table}[!htbp] \centering 
  \caption{\label{ineq1}Determinants of Income Inequality (FE)} 
  \label{} 
\small 
\begin{tabular}{@{\extracolsep{5pt}}lcccccc} 
\\[-1.8ex]\hline 
\hline \\[-1.8ex] 
 & \multicolumn{6}{c}{\textit{Dependent variable:}} \\ 
\cline{2-7} 
\\[-1.8ex] & (1) & (2) & (3) & (4) & (5) & (6)\\ 
\hline \\[-1.8ex] 
 IRR & 0.191$^{***}$ & 0.700$^{***}$ & 0.196$^{**}$ & 1.012$^{***}$ & 0.843$^{**}$ & 5.226$^{***}$ \\ 
  & (0.062) & (0.046) & (0.085) & (0.089) & (0.414) & (0.748) \\ 
  Unemployment & $-$0.100$^{***}$ & $-$0.098$^{***}$ & $-$0.160$^{***}$ & $-$0.147$^{***}$ & $-$0.586$^{***}$ & $-$0.814 \\ 
  & (0.017) & (0.017) & (0.025) & (0.040) & (0.210) & (0.541) \\ 
  Inflation & $-$0.087$^{***}$ & 0.040$^{*}$ & $-$0.128$^{***}$ & 0.155$^{***}$ & $-$0.680$^{***}$ & 0.548$^{**}$ \\ 
  & (0.010) & (0.021) & (0.013) & (0.043) & (0.082) & (0.267) \\ 
  Union Density & $-$0.091$^{***}$ & $-$0.088$^{***}$ & $-$0.159$^{***}$ & $-$0.196$^{***}$ & $-$0.917$^{***}$ & $-$1.180$^{***}$ \\ 
  & (0.008) & (0.013) & (0.011) & (0.025) & (0.063) & (0.132) \\ 
  GDP Growth & 0.044 & 0.080$^{***}$ & 0.064 & 0.167$^{***}$ & 0.536$^{*}$ & 0.871$^{**}$ \\ 
  & (0.033) & (0.021) & (0.049) & (0.039) & (0.277) & (0.399) \\ 
  Human Capital &  & 3.556$^{***}$ &  & 7.750$^{***}$ &  & 32.537$^{***}$ \\ 
  &  & (0.705) &  & (1.225) &  & (8.559) \\ 
  Financial Assets &  & 0.204$^{***}$ &  & 0.240$^{***}$ &  & 1.346$^{**}$ \\ 
  &  & (0.038) &  & (0.068) &  & (0.552) \\ 
  Trade &  & $-$0.011 &  & $-$0.017 &  & $-$0.036 \\ 
  &  & (0.007) &  & (0.013) &  & (0.089) \\ 
  Welfare &  & 0.077$^{**}$ &  & 0.213$^{***}$ &  & 0.810$^{**}$ \\ 
  &  & (0.031) &  & (0.058) &  & (0.353) \\ 
  Left Cabinet &  & 0.218$^{*}$ &  & 0.475$^{**}$ &  & 3.951$^{**}$ \\ 
  &  & (0.123) &  & (0.199) &  & (1.601) \\ 
  Constant & 12.251$^{***}$ & $-$4.651$^{**}$ & 17.758$^{***}$ & $-$16.574$^{***}$ & 102.510$^{***}$ & $-$43.905 \\ 
  & (0.403) & (2.269) & (0.572) & (3.890) & (3.103) & (27.522) \\ 
 \hline \\[-1.8ex] 
Observations & 768 & 411 & 768 & 411 & 768 & 411 \\ 
R$^{2}$ & 0.737 & 0.870 & 0.766 & 0.870 & 0.720 & 0.800 \\ 
Adjusted R$^{2}$ & 0.727 & 0.863 & 0.758 & 0.862 & 0.710 & 0.788 \\ 
\hline 
\hline \\[-1.8ex] 
\textit{Note:}  & \multicolumn{6}{r}{$^{*}$p$<$0.1; $^{**}$p$<$0.05; $^{***}$p$<$0.01} \\ 
\end{tabular} 
\end{table}

\begin{table}[!htbp] \centering 
  \caption{\label{ineq2}Determinants of Income Inequality (Pooled)} 
  \label{} 
\small 
\begin{tabular}{@{\extracolsep{5pt}}lcccccc} 
\\[-1.8ex]\hline 
\hline \\[-1.8ex] 
 & \multicolumn{6}{c}{\textit{Dependent variable:}} \\ 
\cline{2-7} 
\\[-1.8ex] & (1) & (2) & (3) & (4) & (5) & (6)\\ 
\hline \\[-1.8ex] 
 IRR & 0.152$^{***}$ & 0.133$^{***}$ & 0.226$^{***}$ & 0.113$^{*}$ & 1.171$^{***}$ & 0.397 \\ 
  & (0.019) & (0.031) & (0.031) & (0.066) & (0.188) & (0.298) \\ 
  Unemployment & $-$0.126$^{***}$ & 0.142$^{***}$ & $-$0.214$^{***}$ & 0.256$^{***}$ & $-$0.526$^{***}$ & 1.741$^{***}$ \\ 
  & (0.014) & (0.029) & (0.024) & (0.055) & (0.171) & (0.520) \\ 
  Inflation & $-$0.105$^{***}$ & 0.010 & $-$0.149$^{***}$ & 0.068 & $-$0.592$^{***}$ & $-$0.066 \\ 
  & (0.010) & (0.028) & (0.016) & (0.054) & (0.124) & (0.329) \\ 
  Union Density & $-$0.051$^{***}$ & $-$0.026$^{***}$ & $-$0.104$^{***}$ & $-$0.058$^{***}$ & $-$0.615$^{***}$ & $-$0.359$^{***}$ \\ 
  & (0.003) & (0.006) & (0.004) & (0.011) & (0.023) & (0.068) \\ 
  GDP Growth & 0.047 & 0.220$^{***}$ & 0.050 & 0.360$^{***}$ & 0.434 & 1.724$^{***}$ \\ 
  & (0.040) & (0.037) & (0.068) & (0.070) & (0.353) & (0.651) \\ 
  Human Capital &  & 5.131$^{***}$ &  & 9.339$^{***}$ &  & 46.858$^{***}$ \\ 
  &  & (0.361) &  & (0.714) &  & (4.616) \\ 
  Financial Assets &  & 0.081$^{***}$ &  & 0.232$^{***}$ &  & 1.052$^{***}$ \\ 
  &  & (0.025) &  & (0.051) &  & (0.340) \\ 
  Trade &  & $-$0.044$^{***}$ &  & $-$0.086$^{***}$ &  & $-$0.403$^{***}$ \\ 
  &  & (0.003) &  & (0.004) &  & (0.036) \\ 
  Welfare &  & 0.063$^{**}$ &  & $-$0.026 &  & $-$1.095$^{*}$ \\ 
  &  & (0.026) &  & (0.060) &  & (0.592) \\ 
  Left Cabinet &  & 0.221 &  & $-$0.096 &  & 3.598 \\ 
  &  & (0.164) &  & (0.310) &  & (2.311) \\ 
  Constant & 12.364$^{***}$ & $-$6.208$^{***}$ & 17.878$^{***}$ & $-$13.591$^{***}$ & 88.770$^{***}$ & $-$54.934$^{***}$ \\ 
  & (0.199) & (1.142) & (0.415) & (2.042) & (1.870) & (15.648) \\ 
 \hline \\[-1.8ex] 
Observations & 768 & 411 & 768 & 411 & 768 & 411 \\ 
R$^{2}$ & 0.257 & 0.543 & 0.271 & 0.586 & 0.275 & 0.548 \\ 
Adjusted R$^{2}$ & 0.253 & 0.532 & 0.266 & 0.576 & 0.270 & 0.536 \\ 
\hline 
\hline \\[-1.8ex] 
\textit{Note:}  & \multicolumn{6}{r}{$^{*}$p$<$0.1; $^{**}$p$<$0.05; $^{***}$p$<$0.01} \\ 
\end{tabular} 
\end{table}

\hypertarget{conclusion}{%
\section{Conclusion}\label{conclusion}}

Binder fails to refute any of King's criticisms. Instead, she reinforces
there validity, either by intentionally refusing or simply failing, to
provide any comprehensive counter-arguments, only ones that touch on one
particular aspect of the original criticism.

\clearpage
\newpage
\onehalfspace
\fancypagestyle{BIB}{%
    \fancyhead[R]{}
    \fancyhead[L]{}
}

\thispagestyle{BIB}

\hypertarget{bibliography}{%
\section*{Bibliography}\label{bibliography}}
\addcontentsline{toc}{section}{Bibliography}

\hypertarget{refs}{}
\begin{CSLReferences}{1}{0}
\leavevmode\vadjust pre{\hypertarget{ref-binder2018c}{}}%
Binder, Sarah, Mark Spindel, Desmond King, \& Lawrence R. Jacobs. 2018.
{``Critical Dialogue.''} \emph{Perspectives on Politics} 16(3): 781--83.

\leavevmode\vadjust pre{\hypertarget{ref-boushey2018}{}}%
Boushey, Heather, J. Bradford DeLong, \& Marshall Steinbaum, eds. 2018.
\emph{After Piketty: The Agenda for Economics and Inequality}.
Cambridge, MA: Harvard University Press.

\leavevmode\vadjust pre{\hypertarget{ref-dorr2018a}{}}%
Dörr, Patricia. 2018. \emph{The Impact of Monetary Policy on Economic
Inequality}. Wiesbaden: Springer International Publishing.

\leavevmode\vadjust pre{\hypertarget{ref-jacobs2016}{}}%
Jacobs, Lawrence R., \& Desmond S. King. 2016. \emph{Fed Power: How
Finance Wins}. New York: Oxford University Press.

\leavevmode\vadjust pre{\hypertarget{ref-king2018b}{}}%
King, Desmond. 2018. {``The Myth of Independence: How Congress Governs
the Federal Reserve. By Sarah Binder and Mark Spindel. Princeton:
Princeton University Press, 2017. 296p. \$35.00 Cloth.''}
\emph{Perspectives on Politics} 16(3): 779--81.

\leavevmode\vadjust pre{\hypertarget{ref-lapavitsas2013c}{}}%
Lapavitsas, Costas. 2013. \emph{Profiting Without Producing: How Finance
Exploits Us All}. London: Verso.

\leavevmode\vadjust pre{\hypertarget{ref-leonard2022}{}}%
Leonard, Christopher. 2022. \emph{The Lords of Easy Money}. New York:
Simon \& Schuster.

\leavevmode\vadjust pre{\hypertarget{ref-piketty2017}{}}%
Piketty, Thomas. 2017. \emph{Capital in the Twenty-First Century}.
Cambridge, MA: The Belknap Press of Harvard University Press.

\leavevmode\vadjust pre{\hypertarget{ref-piketty2020}{}}%
---------. 2020. \emph{Capital and Ideology}. Cambridge, MA: The Belknap
Press of Harvard University Press.

\leavevmode\vadjust pre{\hypertarget{ref-piketty2022}{}}%
---------. 2022. \emph{A Brief History of Equality}. Cambridge, MA: The
Belknap Press of Harvard University Press.

\leavevmode\vadjust pre{\hypertarget{ref-spindel2017}{}}%
Spindel, Mark, \& Sarah Binder. 2017. \emph{The Myth of Independence:
How Congress Governs the Federal Reserve}. Princeton: Princeton
University Press.

\leavevmode\vadjust pre{\hypertarget{ref-toussaint2019}{}}%
Toussaint, Eric. 2019. \emph{The Debt System: A History of Sovereign
Debts and Their Repudiation}. Chicago: Haymarket Books.

\leavevmode\vadjust pre{\hypertarget{ref-worldbank2022}{}}%
World Bank. 2022. {``Chapter 4: Calculating Survey Estimates of Poverty
and Inequality.''} In \emph{Poverty and Inequality Platform Methodology
Handbook}, World Bank.

\end{CSLReferences}

\singlespace
\newpage
\fancypagestyle{APP}{%
    \fancyhead[R]{}
    \fancyhead[L]{}
}

\thispagestyle{APP}

\hypertarget{appendix}{%
\section*{Appendix}\label{appendix}}
\addcontentsline{toc}{section}{Appendix}

\end{document}
